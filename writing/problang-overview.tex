\documentclass{sp}

% The \pdf* commands provide metadata for the PDF output.
% Do not use LaTeX style / commands like \emph{} inside these.
\pdfauthor{Author Full Name(s)}
\pdftitle{Full title}
\pdfkeywords{XXX, XXX}

% Optional short title inside square brackets, for the running headers.
% If no short title is given, no title appears in the headers.
%\title[Adjective ordering preferences]{On the grammatical source of adjective ordering preferences%
\title[Practical introduction to RSA]{A practical introduction to the Rational Speech Act modeling framework%
	\thanks{We thank XXX.}}

% Optional short author inside square brackets, for the running headers.
% If no short author is given, no authors print in the headers.
\author[]{% As many authors as you like, each separated by \AND. %%% uncomment to de-anonymize
%	\spauthor{author1 \\ \institute{affiliation1}} \AND
%	\spauthor{author2 \\ \institute{affiliation2}} \AND
%	\spauthor{author3 \\ \institute{affiliation3}}
}

	

\usepackage{linguex}
\usepackage{qtree}
\qtreecenterfalse
\usepackage{tree-dvips}
\usepackage{phonetic}
\usepackage{xcolor}
\usepackage{pifont}
\usepackage{lineno}
\usepackage{graphicx}
%\usepackage{qdotbranch}
\usepackage{booktabs}
\usepackage{multirow}
\usepackage{CJK}
\usepackage{tikz}
\usepackage{amsfonts} 
\usepackage{amsmath}
\usepackage{hyperref}

\def\url#1{\expandafter\string\csname #1\endcsname}

\newcommand{\gcs}[1]{\textcolor{blue}{[gcs: #1]}}  

\newcommand{\type}[1]{\ensuremath{\left \langle #1 \right \rangle }}
\newcommand{\lam}{\ensuremath{\lambda}}
\newcommand{\sem}[1]{\mbox{$[\![$#1$]\!]$}}

\renewcommand{\firstrefdash}{}

\begin{document}

\maketitle

\begin{abstract}
	Recent advances in computational cognitive science (i.e., simulation-based probabilistic programs) have paved the way for significant progress in formal, implementable models of pragmatics. Rather than describing a pragmatic reasoning process, these models articulate and implement one, deriving both qualitative and quantitative predictions of human behavior---predictions that consistently prove correct, demonstrating the viability and value of the framework. The current paper provides a practical introduction to the Bayesian Rational Speech Act modeling framework, serving as a companion piece to the hands-on web-book at \href{https://www.problang.org}{www.problang.org}. After providing a conceptual overview of the framework, we then walk readers through the resources of the web-book.
\end{abstract}

\begin{keywords}
	XXX, XXX
\end{keywords}

\section{Introduction}

Goodies list of RSA

\section{High-level overview of framework}

The RSA framework views language understanding as a process of recursive social reasoning between speakers and listeners: listeners interpret the utterances they hear by reasoning about how speakers generate them; speakers choose their utterances by reasoning about how listeners interpret them. In the basic, vanilla RSA model, this recursion involves three layers of inference. Typically formulated as statements of conditional probability, as in (\ref{L0}--\ref{L1}), these inference layers correspond to models of speakers and listeners.

\begin{equation} \label{L0}
P_{L_0}(s|u) \propto \delta_{[\![u]\!](s)}
\end{equation}
\begin{equation} \label{U}
U_{S_1}(u; s) = \textrm{log}P_{L_0}(s|u) - C(u)
\end{equation}
\begin{equation} \label{S1}
P_{S_1}(u|s) \propto \textrm{exp}(\alpha \cdot U_{S_1}(u;s))
\end{equation}
\begin{equation} \label{L1}
P_{L_1}(s|u) \propto P_{S_1}(u|s) \cdot P(s)
\end{equation}

The reasoning grounds out in the naive, literal listener, $L_0$, who interprets utterances according to their literal semantics. In other words, $L_0$ hears some utterance $u$ and infers the state of the world $s$ that $u$ was meant to describe. $L_0$ performs this inference by restricting the set of possible states to just those that are compatible with the literal, truth-functional semantics of $u$, returning a uniform probability distribution over the states $s$ that $u$ maps to \texttt{true}.

One layer up, a pragmatic speaker, $S_1$, chooses utterances in proportion to their utility $U_{S_{1}}$. Utterances are useful to the extent that they maximize the probability that $L_0$ will infer the correct $s$ on the basis of $u$, while minimizing the cost of $u$ (speakers aim to be efficient). So, when selecting utterances, $S_1$ considers their affect on interpretation (i.e., on $L_0$'s resulting beliefs).

At the top layer of inference, the pragmatic listener, $L_1$, interprets utterances to infer the true state of the world. However, unlike $L_0$, who reasons directly about the utterance semantics, $L_1$ reasons instead about the process that generated the utterance; that process is the speaker $S_1$. $L_1$ thus infers $s$ on the basis of $u$ by reasoning about the probability that $S_1$ would have chosen $u$ to signal $s$ to $L_0$; the higher that probability, the more likely $L_1$ is to conclude that $S_1$ indeed intended to communicate $s$. Because $L_1$ reasons about $S_1$, who in turn reasons about the literal semantics in $L_0$, $L_1$'s interpretation is affected by the semantics of $u$, albeit only indirectly via the $S_1$ layer. This space between the semantics (i.e., $L_0$) and the resulting interpretation (i.e., the posterior beliefs of $L_1$) is where pragmatics enters.

MF's math (derivable from KL, connection to Grice maxims)

\section{Variations on vanilla}

\subsection{Lexical inference type models}

write like this, useful for this, have the following goodies..

\subsection{QUD inference}

\subsection{Context/Prior inference}

\subsection{Epistemic inference}

\subsection{Complex utility/utilty inference}

\section{Modeling practicalities}

Free parameters (optimality, cost, alternatives)

World priors: Both how to model and how to measure

QUD

Linking functions

\section{Extensions/limitations}

Incrementality / processing

Compositionality (CCG chapter in dippl?)

NLP (Dan Klein)

Individual variability (cite Franke \& Degen)

``Just'' a computational level theory. Doesn't bear on mechanisms. But, potentially a hook via resource rational analysis (Leider, Griffiths, ...)

Hand-coded: Don't have a theory of alternatives, state priors

\section{Summary and outlook}

XXX


%\bibliography{problang}

%\begin{addresses} %%% uncomment to de-anonymize
%	\begin{address}
%		author1
%		\email{email1}
%	\end{address}
%	\begin{address}
%		author2
%		\email{email2}
%	\end{address}
%	\begin{address}
%		author3
%		\email{email3}
%	\end{address}
%\end{addresses}



\end{document}
