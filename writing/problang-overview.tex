\documentclass{sp}

% The \pdf* commands provide metadata for the PDF output.
% Do not use LaTeX style / commands like \emph{} inside these.
\pdfauthor{Author Full Name(s)}
\pdftitle{Full title}
\pdfkeywords{XXX, XXX}

% Optional short title inside square brackets, for the running headers.
% If no short title is given, no title appears in the headers.
%\title[Adjective ordering preferences]{On the grammatical source of adjective ordering preferences%
\title[Practical introduction to RSA]{A practical introduction to the Rational Speech Act modeling framework%
	\thanks{We thank XXX.}}

% Optional short author inside square brackets, for the running headers.
% If no short author is given, no authors print in the headers.
%\author[shortauthor]{% As many authors as you like, each separated by \AND. %%% uncomment to de-anonymize
%	\spauthor{author1 \\ \institute{affiliation1}} \AND
%	\spauthor{author2 \\ \institute{affiliation2}} \AND
%	\spauthor{author3 \\ \institute{affiliation3}}}

	

\usepackage{linguex}
\usepackage{qtree}
\qtreecenterfalse
\usepackage{tree-dvips}
\usepackage{phonetic}
\usepackage{xcolor}
\usepackage{pifont}
\usepackage{lineno}
\usepackage{graphicx}
%\usepackage{qdotbranch}
\usepackage{booktabs}
\usepackage{multirow}
\usepackage{CJK}
\usepackage{tikz}
\usepackage{amsfonts} 
\usepackage{amsmath}
\usepackage{hyperref}

\def\url#1{\expandafter\string\csname #1\endcsname}

\newcommand{\gcs}[1]{\textcolor{blue}{[gcs: #1]}}  

\newcommand{\type}[1]{\ensuremath{\left \langle #1 \right \rangle }}
\newcommand{\lam}{\ensuremath{\lambda}}

\renewcommand{\firstrefdash}{}

\begin{document}

\maketitle

\begin{abstract}
	Recent advances in computational cognitive science (i.e., simulation-based probabilistic programs) have paved the way for significant progress in formal, implementable models of pragmatics. Rather than describing a pragmatic reasoning process, these models articulate and implement one, deriving both qualitative and quantitative predictions of human behavior---predictions that consistently prove correct, demonstrating the viability and value of the framework. The current paper provides a practical introduction to the Bayesian Rational Speech Act modeling framework, serving as a companion piece to the hands-on web-book at \href{https://www.problang.org}{www.problang.org}. After providing a conceptual overview of the framework, we then walk readers through the resources of the web-book.
\end{abstract}

\begin{keywords}
	XXX, XXX
\end{keywords}

\section{Introduction}

XXX

\section{Conceptual overview of the RSA framework}

XXX

\section{Chapter overview}

XXX

\section{Summary and outlook}

XXX


\bibliography{problang}

%\begin{addresses} %%% uncomment to de-anonymize
%	\begin{address}
%		author1
%		\email{email1}
%	\end{address}
%	\begin{address}
%		author2
%		\email{email2}
%	\end{address}
%	\begin{address}
%		author3
%		\email{email3}
%	\end{address}
%\end{addresses}



\end{document}